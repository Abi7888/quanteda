
%this stops output over-running the shaded output area in the pdf 




\documentclass{article}\usepackage[]{graphicx}\usepackage[]{color}
%% maxwidth is the original width if it is less than linewidth
%% otherwise use linewidth (to make sure the graphics do not exceed the margin)
\makeatletter
\def\maxwidth{ %
  \ifdim\Gin@nat@width>\linewidth
    \linewidth
  \else
    \Gin@nat@width
  \fi
}
\makeatother

\definecolor{fgcolor}{rgb}{0.345, 0.345, 0.345}
\newcommand{\hlnum}[1]{\textcolor[rgb]{0.686,0.059,0.569}{#1}}%
\newcommand{\hlstr}[1]{\textcolor[rgb]{0.192,0.494,0.8}{#1}}%
\newcommand{\hlcom}[1]{\textcolor[rgb]{0.678,0.584,0.686}{\textit{#1}}}%
\newcommand{\hlopt}[1]{\textcolor[rgb]{0,0,0}{#1}}%
\newcommand{\hlstd}[1]{\textcolor[rgb]{0.345,0.345,0.345}{#1}}%
\newcommand{\hlkwa}[1]{\textcolor[rgb]{0.161,0.373,0.58}{\textbf{#1}}}%
\newcommand{\hlkwb}[1]{\textcolor[rgb]{0.69,0.353,0.396}{#1}}%
\newcommand{\hlkwc}[1]{\textcolor[rgb]{0.333,0.667,0.333}{#1}}%
\newcommand{\hlkwd}[1]{\textcolor[rgb]{0.737,0.353,0.396}{\textbf{#1}}}%

\usepackage{framed}
\makeatletter
\newenvironment{kframe}{%
 \def\at@end@of@kframe{}%
 \ifinner\ifhmode%
  \def\at@end@of@kframe{\end{minipage}}%
  \begin{minipage}{\columnwidth}%
 \fi\fi%
 \def\FrameCommand##1{\hskip\@totalleftmargin \hskip-\fboxsep
 \colorbox{shadecolor}{##1}\hskip-\fboxsep
     % There is no \\@totalrightmargin, so:
     \hskip-\linewidth \hskip-\@totalleftmargin \hskip\columnwidth}%
 \MakeFramed {\advance\hsize-\width
   \@totalleftmargin\z@ \linewidth\hsize
   \@setminipage}}%
 {\par\unskip\endMakeFramed%
 \at@end@of@kframe}
\makeatother

\definecolor{shadecolor}{rgb}{.97, .97, .97}
\definecolor{messagecolor}{rgb}{0, 0, 0}
\definecolor{warningcolor}{rgb}{1, 0, 1}
\definecolor{errorcolor}{rgb}{1, 0, 0}
\newenvironment{knitrout}{}{} % an empty environment to be redefined in TeX

\usepackage{alltt}
\title{Introduction to text mining Twitter with R}
\author{Paul Nulty}
\usepackage{hyperref}
\IfFileExists{upquote.sty}{\usepackage{upquote}}{}
\begin{document}
\maketitle
\clearpage

\section*{List of Acronyms and Definitions}
\begin{itemize}
  \item API: Application Programming Interface --- a specification of how computer programs can communicate with each other
  \item JSON: JavaScript Object Notation --- a human readable format for representing attribute-value pairs.
  \item OAuth: An open standard for authorization. OAuth provides client applications a `secure delegated access' to server resources on behalf of a resource owner.
  \item HTTP: HyperText Transfer Protocol: 
  \item REST: REpresentational State Transfer:
\end{itemize}

\section*{Useful Links}
\url{https://dev.twitter.com/docs/things-every-developer-should-know}

\clearpage


\begin{knitrout}
\definecolor{shadecolor}{rgb}{0.969, 0.969, 0.969}\color{fgcolor}\begin{kframe}
\begin{alltt}
\hlnum{2} \hlopt{+} \hlnum{3}
\end{alltt}
\begin{verbatim}
## [1] 5
\end{verbatim}
\begin{alltt}
\hlkwd{library}\hlstd{(quanteda)}
\hlkwd{tokenize}\hlstd{(}\hlstr{"Just testing sweave/knitr code integration here."}\hlstd{)}
\end{alltt}
\begin{verbatim}
## [1] "just"        "testing"     "sweaveknitr" "code"       
## [5] "integration" "here"
\end{verbatim}
\end{kframe}
\end{knitrout}



\section*{The twitter APIs}
Twitter offers two ways for computer programs to post and retrieve data from its service. The twitter REST Search API will receive requests as http commands, and will respond with a JSON object representing a list of tweets. The twitter streaming API allows a computer program to maintain a connection to twitter's service and receive a flow of live public tweets that match the filters or search terms specified.

Both of these APIs require that the developer of the computer program create a twitter account, create a new application, and receive an access token.

\section*{Getting programmatic access to twitter}
\begin{enumerate}
  \item Go to https://dev.twitter.com/ and sign in with a twitter account
  \item Go to My Applications and select "New App". https://apps.twitter.com/
  \item Fill out the form and click 'create'
\end{enumerate}
\end{document}
